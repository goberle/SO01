\documentclass{report}
\usepackage[utf8]{inputenc}
\usepackage[T1]{fontenc}
\usepackage[francais]{babel}
\usepackage[final]{pdfpages}

\title{Sociologie du travail\\Comment la souffrance au travail a-t-elle évoluée du capitalisme industriel au capitalisme financier ?}
\author{Vincent Chabrette\\Guillaume Oberlé}
\date{18 Décembre 2013}

\begin{document}

\makeatletter
\def\clap#1{\hbox to 0pt{\hss #1\hss}}%
\def\ligne#1{%
\hbox to \hsize{%
\vbox{\centering #1}}}%
\def\haut#1#2#3{%
\hbox to \hsize{%
\rlap{\vtop{\raggedright #1}}%
\hss
\clap{\vtop{\centering #2}}%
\hss
\llap{\vtop{\raggedleft #3}}}}%
\def\bas#1#2#3{%
\hbox to \hsize{%
\rlap{\vbox{\raggedright #1}}%
\hss
\clap{\vbox{\centering #2}}%
\hss
\llap{\vbox{\raggedleft #3}}}}%

\def\maketitle{%
\thispagestyle{empty}\vbox to \vsize{%
\haut{}{\@blurb}{}
\vfill
\vspace{1cm}
\begin{flushleft}
\usefont{OT1}{ptm}{m}{n}
\huge \@title
\end{flushleft}
\hrule height 4pt
\begin{flushright}
\usefont{OT1}{phv}{m}{n}
\large \@author
\end{flushright}
\begin{flushleft}
\end{flushleft}
\vspace{1cm}
\vfill
\begin{center}
\includegraphics[width=.5\textwidth]{./logo.png}
\end{center}
\vfill
\vfill
\bas{}{\@location, le \@date}{}
}%
\cleardoublepage
}
\def\date#1{\def\@date{#1}}
\def\author#1{\def\@author{#1}}
\def\title#1{\def\@title{#1}}
\def\location#1{\def\@location{#1}}
\def\blurb#1{\def\@blurb{#1}}
\makeatother

\title{Comment la souffrance au travail a-t-elle évoluée du capitalisme industriel au capitalisme financier ?}
\author{Vincent \textsc{Chabrette}\\Guillaume \textsc{Oberlé}}
\location{Belfort}
\blurb{%
Sociologie du travail\\Université de Technologie de Belfort-Montbeliard}
\maketitle
\tableofcontents

\chapter*{Introduction}
\addcontentsline{toc}{chapter}{Introduction}
	\paragraph{}
		Dans le cadre de l'UV SO01, Sociologie du travail, nous avons choisi, pour l'évaluation de la matière, de traiter un sujet sur le thème de la souffrance au travail. Nous avons décidé de nous intéresser plus particulièrement aux évolutions de la souffrance au travail. Ce rapport aura donc pour objectif de répondre à la problématique \textit{Comment la souffrance au travail a-t-elle évoluée du capitalisme industriel au capitalisme financier ?}

	\paragraph{}
		L’objectif sera d’analyser les différences entre la souffrance au travail dans le cadre du capitalisme industriel des années 60 et celle dans le cadre du capitalisme financier d’aujourd’hui. \\Ce sujet nous a été inspiré par le film ``J’ai (très) mal au travail'' de Jean-Michel Carré qui constituera aussi une de nos sources principales d’information.

	\paragraph{}
		Pour répondre à cette problématique, nous avons divisé notre rapport en quatre grandes parties. Tout d'abord nous nous intéresserons à la souffrance dans le cadre industriel. Nous essayerons de montrer qu'elle se traduit inéluctablement par une souffrance physique, mais aussi par une souffrance psychique non négligeable. Dans une deuxième partie, nous étudierons les différents aspects de la souffrance au travail dans notre société actuelle, majoritairement dominée par le capitalisme financier. Puis, dans une troisième partie, nous nous poserons la question de l'acceptation et de la banalisation de la souffrance au travail. Enfin, dans une quatrième et dernière partie, nous énoncerons ce qu'il est possible de faire, et ce qui fait à l'heure actuelle, pour contrer et traiter la souffrance au travail.

\chapter{La souffrance dans le cadre industriel : une souffrance physique ?}
	\section{Les manifestations de la souffrance au travail}
		\paragraph{}
			Le capitalisme industriel, qui a vu son apparition au début des années 1960 se caractérise par un productivisme nécessitant beaucoup d'ouvriers travaillant à la chaîne. En effet, c'est dans ces années là que nos sociétés modernes ont vu apparaître de nouvelles formes de consommations basé sur la création perpétuelle de nouveau besoin. Ces besoins créé par les grandes entreprises nécessite donc d'importante force de travail pour permettre une production de masse, ce qui impose aux chaînes de production des cadences parfois inhumaine, notamment du à l'organisation scientifique du travail (Taylorisme/Fordisme).

		\paragraph{}
			De ce fait, le capitalisme industriel est fortement créateur chez ces travailleurs de troubles musculo-squelettique, d'accident du travail et autres diverses pathologie physiques.\\
			Ces souffrances physiques sont ... imputables aux gestes répétés, aux cadences de travail, mais aussi assez souvent à cause du salarié lui-même qui va être amené à négliger les règles de sécurité pour pouvoir répondre à la pression qui leur est imposé. 

	\newpage{}
	\section{Une souffrance psychique non négligeable}
		\paragraph{}
			Même si la souffrance physique est la première venant à l'esprit lorsque l'on parle de capitalisme industriel, celui-ci est aussi créateur de souffrance psychique. À l'instar de la souffrance physique, la souffrance psychique est ... aux cadences et à la pression imposé par le travail ouvrier. A noter aussi qu'une partie de celle-ci est imputable à la souffrance physique. Cependant une grande partie de cette souffrance psychique était contre balancé par l'esprit de camaraderie, les dirigeants paternalistes, etc...

		\paragraph{}
			En revanche depuis la perte de vitesse du capitalisme industriel face au capitalisme financier, à cause de la concurrence omniprésente ainsi que de l'instabilité du marché du travail, le travail ouvrier tend à partager avec les grandes entreprises de services certains de leur éléments de souffrance psychique. En effet, les relations au travail se détériorent aussi et l'ouvrier vie dans la peur constante de l'avenir qui est, comme la conservation de leur emploi, baigné d'incertitudes.

\chapter{La souffrance au travail dans notre société actuelle}
	\paragraph{}
		\textit{``Chaque année 2 millions de salariés subissent maltraitance et harcèlement moral''} - Tiré du film ``J’ai (très) mal au travail'' de Jean-Michel Carré.
	\section{La souffrance au travail dans le cadre du capitalisme financier}
		\paragraph{}
			Dans nos sociétés modernes soumises à la mondialisation, le marché du travail des pays développés a vu l'importance des secteurs primaires et secondaires diminuer face au secteur tertiaire. Ajouté à cela des sociétés toujours plus basées sur la consommation, on se rend alors compte que cela crée un capitalisme principalement tourné vers le profit : c'est ce que l'on appelle communément le capitalisme financier. Il se différencie du capitalisme industriel par le fait qu'il crée énormément de richesse avec peu de moyens humains là ou ce dernier créé un peu moins de richesses en nécessitant beaucoup plus de travailleurs.
		\paragraph{}
			Il parait évident que ce capitalisme n'est pas principalement créateur de souffrance physique. En revanche, la recherche du profit est fortement créatrice de souffrance psychique. En effet, cette course au profit implique une modification profonde du rapport et de l'organisation au travail.

	\section{Des outils de gestions créateurs de souffrance psychique}
		\paragraph{}
			De part la forte concurrence au sein du monde du travail, l'entreprise se doit d'être réactive et rapide. Le manager doit être en mesure de chiffrer les résultats des personnes qu'il encadre et répercute la pression qu'il reçoit aux salariés dont il a la charge; en partie involontairement, mais aussi et surtout car le management par la pression et la peur est une méthode managériale qui fut longtemps prônée comme la plus efficace. Cette technique est donc logiquement fortement créatrice de stress et de mal-être chez le salarié.
		\paragraph{}
		 	Ainsi, pour "aider" l'employé à faire face à cette pression, on assiste de plus en plus à la création de séminaires, notamment pour les cadres, visant à les remotiver, à augmenter la cohésion d'équipe, à gérer leur stress, etc... Leurs employeurs s'assurent ainsi (et surtout) de créer une culture d'entreprise et de lier les salariés entre eux, les impliquant encore plus dans la société. Certaines entreprises vont, par exemple, embaucher des entraîneurs de rugby qui organiseront des activités ludiques durant lesquelles la hiérarchie n'a plus d'importance (ou n'est, du moins, plus censée) et où seul l'objectif imposé compte, exercice étant censé rapprocher les managers des managés, mais qui au final visera surtout à humaniser les relations entre eux pour que le managé accepte plus facilement ce que lui demande son manager. 
		\paragraph{}
			L'arrivée des nouvelles technologies de l'information et de la communication a elle aussi permit aux managers de trouver de nouvelles façons de conserver constamment les salariés au service de l'entreprise en créant un lien permanent entre l'employé et son employeur. \\La démocratisation de la téléphonie mobile et de l'Internet fit entrer le monde de l'entreprise dans une époque de l'urgence et de l'instantanéité. Ainsi, les employeurs n'hésitent plus à fournir un téléphone mobile ainsi qu'un ordinateur portable à chaque employé. L'employé doit alors être disponible, en tout temps et à toute heure, pour son entreprise.
		\paragraph{}
			La notation des employés est également un outil de gestion supplémentaire qui s'offre au manager. La technique communément utilisée est celle de l'autoévaluation où l'on va demander à l'employé sont avis sur la notation que vient de faire son supérieur, et ainsi permettre de faire participer le salarié à son évaluation de sorte qu'il accepte plus facilement les conséquences, qu'elles soit positives ou négatives.
		\paragraph{}
			Ces techniques managériales ne sont que des exemples parmi tant d'autres, cependant la perversité de ces nouvelles formes de management commence dès l'embauche; en effet, on n'embauche plus les gens uniquement sur leurs compétences mais aussi sur leur capacité à se soumettre à la hiérarchie et à rentrer facilement dans le moule de l'entreprise. On va alors privilégier des personnes ayant beaucoup à perdre s'il perdaient leur emplois (personne avec famille, prêts à rembourser, etc...)
		\paragraph{}
			Via ce nouveau management, l'entreprise prend de plus en plus le pas sur la vie privée des salariés : l'employé ne peut alors cesser de "penser entreprise" que cela soit chez lui ou même lors de la pause café. Cela ancre alors profondément l'entreprise dans la vie de la personne.

		%- concurrence forte au sein du monde du travail
		%  -> l'entreprise se doit d'être réactive, rapide
		%  -> on demande au manager plus de résultat
		%  -> manager répercute la pression sur les employés
		%- nouvelle technologie, l'employé est lié à l'entreprise
		%  -> époque de l'urgence et de l'instantanéité
		%- on embauche des gens qui se soumettent au détriment des compétences
		%  -> gens qui ont des choses à perdre
		%- organisation de séminaire pour créer une culture d'entreprise et lié les salariés entre eux
		%- notation des employées
		%  -> autoévaluation : implique dans la notation et lui fait accepter plus facilement les conséquences positives et négative
		%l'entreprise percute la vie privé des salariés
		%  -> les gens pense en permanence entreprise (pause café, soir, etc..)
		%- beauté du geste : salarié expérimenté en face de nouveau
		%* But: impliquer les salariés dans l'entreprise et dans leur travail (en va de la vie de l'entreprise, cf: Renault)		

	\section{La productivité au détriment de l'humain}
		\paragraph{}
			Pour survivre dans un marché hautement concurrentiel, l'entreprise va, en plus d'opérer une refonte de son management (comme vu précédemment), chercher tous les moyens possibles pour augmenter sa productivité. 
		\paragraph{}
			L'entreprise peut alors opérer des changements de postes par exemple. Ainsi, dans le film "J'ai (très) mal au travail", nous avons pu découvrir le cas d'une ouvrière dans le secteur textile qui travaillait dans un atelier de tissage et qui a été déplacée à un autre poste moins valorisant. N'ayant plus le sentiment d'effectuer un travail important et utile, cette ouvrière fut désespérée de sa situation et à en souffrir profondément, la rendant même prête à se faire mal pour pouvoir obtenir un arrêt de travail. On se rend alors compte que cette décision de la direction, prise sans aucune considération de la volonté propre de la salariée, créa elle aussi une véritable souffrance psychique pour elle.
		\paragraph{}
			Avec la mondialisation, les sociétés ont aussi la possibilité de délocaliser leurs usines pour diminuer leurs coûts de production. Il se peut aussi qu'en cas de grande difficulté financière (ou d'un patron abusant du système pour faire un maximum de profit, au détriment de la survie de la société) l'entreprise soit amenée à fermer des lieux de travail, qu'il s'agisse d'usines ou de bureaux. Dans ces 2 cas, il y a un dénominateur commun : le licenciement de masse.\\
			Acte quasi-abstrait pour les dirigeants d'entreprise, ces licenciements marquent en revanche pour de nombreux salariés (qu'il s'agisse de dizaines ou de milliers) le basculement de leur vie. C'est en effet à ce moment là que les (ex-)salariés vont prendre conscience de la place qu'avait prise l'entreprise dans leur vie. Ainsi, lors de fermeture de sites de productions par exemple, nombreuses sont les personnes sans diplôme, d'un âge déjà relativement avancé et ayant consacré des dizaines d'années de leur vie à l'entreprise, qui vont se retrouver sans emploi, mais aussi sans statut social et sans une partie de leur identité. Nombreuses sont alors les remises en question, mais aussi les soucis d'argent dû à la difficulté de retrouver un emploi, qui montrent bien un des grands pouvoirs de l'entreprise : un quasi-pouvoir de vie et de mort sur ses salariés.

		%- dans le secteur industriel pour produire plus à plus bas coût les entreprises délocalise et/ou licencie
		%- exemple : changement de poste au détriment de la satisfaction du salarié
		%- relation de vie et de mort avec l'entreprise
		%- productivité en dépit du rapport humain
		%  -> chacun à son poste
		%  -> pas de rapport humain

	\section{Comparaison avec la souffrance physique}
		\paragraph{}
			%http://www.meteojob.com/emploi-actualites/actualite/158/la-souffrance-psychique-au-travail-un-mal-plus-courant-quon-croit
			%http://www.intefp-sstfp.travail.gouv.fr/datas/files/SSTFP/Souffrance_mentale_au_travail%20INTEFP%202005.pdf => (p7) http://dpr.vchabrette.fr/O3xZ
			%20% des maux

\chapter{L’acceptation de la souffrance : la souffrance banalisée ?}
	\paragraph{}
		Malgré l'importance de toutes ces souffrances, de nombreuses personnes vivent au quotidien avec elles et semblent les accepter. L'objectif de cette partie sera d'essayer de comprendre comment des personnes peuvent supporter ces souffrances, ou du moins les accepter.

	\section{Le rapport du salarié au travail}
		\paragraph{}
			La première clé pour bien comprendre ce phénomène est de prendre conscience du rapport que les travailleurs peuvent avoir avec leur travail.\\
			En effet, quelque soit la difficulté ou la pénibilité de celui-ci, l'employé cherchera toujours la reconnaissance de son travail de par ses pairs et/ou ses supérieurs. Il découle de cela une fierté par rapport à son propre travail, fierté qui élève le travail au dela d'une simple source de rémunération mais au niveau d'une véritable source d'épanouissement personnel, d'intégration et de lien social. C'est aussi un élèment important de l'entreprise duquel va dépendre l'implication de ses salariés : une entreprise où les employés ne chercheraient qu'à faire le strict minimum sans s'impliquer n'est pas viable. Ainsi, on peut citer en exemple une étude de la régie Renault où l'expérience menée à été de demander aux employés de respecter strictement leurs obligations, expérience qui a mené l'usine à cesser de fonctionner au bout de quelques dizaines de minutes (témoignage de Paul Ariès, politologue, dans le film "J'ai (très) mal au travail").\\
			Cependant, un des effets secondaires de cette fierté est que l'employé sera plus enclin à accepter ses souffrances. ...
		%- fierté par rapport a son travail
		%	-> le salarié souffre mais le fait de se sentir impliquer dans son travail et surtout de recevoir de la reconnaissance de ses pairs et/ou de ses supérieurs %font qu'il est fier
		%	-> exemple de cas : ouvrier a la chaîne reconnaisse travail de l'autre
		%	-> reconnaissance est une des chose les plus importantes cherché par le salarié
	\section{La concurrence dans le monde du travail}
		\paragraph{}
			Dans le monde du travail actuel, la conccurence est omniprésente, et ce à toutes les échelles.
		\paragraph{}
			À l'échelle de l'entreprise, la conccurence crée chez le salarié une volonté d'être toujours meilleur que les autres, plus efficace, plus productif, plus flexible... une volonté qui va alors faire passer la souffrance au second plan pour que celle ci ne soit pas un frein pour l'évolution de la personne. De plus, cette conccurence va rajouter de la souffrance à cause des fortes tensions entre salariés et de la pression importante qu'elle produit.
		\paragraph{}
			En dehors de l'entreprise, la conccurence est d'autant plus féroce entre les entreprises, et donc entre les salariés de ces entreprises. Soucieux de conserver leurs emplois, les salariés vont alors tout faire pour être meilleurs que leurs homologues des entreprises conccurentes, et ce même si ces entreprises sont étrangères (par exemple chinoises). On se rend bien compte alors que cette conccurence dûe à la mondialisation peut-être parfois totalement déséquilibrée; en effet, un ouvrier français pourra difficilement égaler un ouvrier chinois en terme de productivité, menant l'ouvrier français a pousser sa dévotion au travail au maximum, sans pouvoir de toute façon atteindre le rendement de l'autre, ne serait-ce qu'à cause des différences sur la législation du travail.
		%- concurrence entre les salariés, si on veut être le meilleur, pas le temps de souffrir
		%- la mondialisation met en concurrence les salariés entre salarié du même pays, voir même du monde 
	\section{La valeur de l'emploi}
		\paragraph{}
			De part la situation économique actuelle en France et dans la majorité des pays développés, le marché du travail est particulièrement incertain. En effet, les opportunités d'embauches sont aujourd'hui beaucoup plus faibles qu'il y a quelques années (à cause du capitalisme financier justement) et avoir un emploi, qui plus est stable et non précaire, est un véritable privilège que tout le monde n'a pas; c'est ainsi 25\% de la population française qui est en non-emploi ou en sous-emploi et qui vit dans la précarité.
		\paragraph{}
			Évoluant dans ce contexte précaire, le salarié va alors tout faire pour conserver son emploi, ce qui implique aussi de se résigner face à ses souffrances. Il ne peut en effet pas se permettre de perdre son travail, celui-ci étant nécessaire à sa survie mais étant aussi un des principaux composants de son identité. De plus, avoir un emploi dans cette conjoncture est source de fierté lui aussi, étant donné que le salarié est conscient qu'il a une chance que tout le monde n'a pas.
	\newpage{}
	\section*{Des esclaves modernes}
		\paragraph{}
			Ce que l'on peut retirer de cette analyse de l'acceptation de la souffrance comme une composante normale du travail est que celle-ci participe à déshumaniser le salarié. L'entreprise va ainsi l'utiliser sans prendre réellement en considération l'importante souffrance psychique que cela lui cause, et l'employé lui même va mettre cette souffrance au second plan et la juger normale. Cela créé une forme de servitude volontaire de la part de l'employé qui se rendra corvéable à merci sans rechigner. \\
			On pourra cependant noter quelques efforts de la part des entreprises depuis quelques années quant à la reconnaissance de ces maux du travail. En revanche, cela ne sera pas forcément fait sans une certaine hypocrisie, les sociétés embauchant par exemple des psychologues du travail, faisant alors passer le message que la souffrance psychologique est normale au travail.

		%- marché du travail incertain = pas sur de retrouver un emploi
		%- peur de perdre notre emploi
		%- impression d'avoir quelque chose que tout le monde n'a pas, fierté d'avoir un travail
		%- risque de tout perdre fait accepter n'importe quoi
		%- l'entreprise représente ta vie
		%- a paris 1 sdf sur 5 a du travail
		%- risque de chomage fait accepter de souffrir / de faire souffrir
		%* Cela crée une déhumanisation de l'homme et une servitude volontaire
		%* Ce besoin d'avoir de l'argent est imposé par la société de consommation
\chapter{La souffrance au travail : une fatalité ?}
	\paragraph{}
		Après avoir constaté l'importance des diverses souffrances que les travailleurs subissent, nous sommes en droit de nous demander si cela doit être considéré comme normal et si la résignation est la seule voie possible.

	\section{La lutte contre la souffrance au travail}
		\paragraph{}
			Fort heureusement, la résignation n'est en fait pas la seule voie possible. En effet, les souffrances au travail ont aujourd'hui leur propre législation, notamment grâce aux récentes prises en considération de la souffrance psychique. Cette législation encadrant la souffrance au travail est présente à plusieurs niveaux, allant de la prévention à l'encadrement du soin de ces souffrances. Cependant, si l'entreprise ne respecte pas ces lois, le salarié doit alors se lancer dans une véritable bataille judiciaire contre l'entreprise, coûteuse, pénible et longue.
		\paragraph{}
			Les salariés ainsi que les syndicats ont aussi un rôle important à jouer dans la lutte contre la souffrance au travail. En effet, les grèves et autres manifestations contre ces souffrances prennent aujourd'hui une place non négligeable dans les diverses luttes syndicales. Cependant, ces actions (notamment celles contre la souffrance psychique) se heurtent régulièrement à diverses difficultés, comme le souligne Christophe Dejours : 
		\paragraph{}
			\textit{
			``En situation de chômage et d’injustice, les travailleurs tentant de lutter par des grèves se heurtent à deux types de 
			difficultés qui ont des incidences importantes sur la mobilisation collective et politique : la culpabilisation par les 
			« autres », les politiciens, les intellectuels, les cadres, qui estiment qu’il s’agit de grève de « nantis » qui de plus, 
			constituerait une menace pour la pérennité des entreprises. En 88/89, les grèves organisées par les cheminots et les 
			enseignants ont été très largement dénoncées y compris par la gauche et ont d’ailleurs, dans une large mesure, échoué 
			pour ce motif. 
			La honte de protester quand d’autres sont beaucoup plus mal lotis : l’injustice sociale concerne les chômeurs et les 
			pauvres, ceux qui ont un emploi et des ressources sont des privilégiés. Lorsqu’on évoque la situation de ceux qui 
			souffrent au travail, on déclenche souvent une réaction d’indignation.''}
		\paragraph{}
			Ainsi, ces actions sont vites considérées comme des plaintes de personnes qui, en fait... n'auraient pas à se plaindre.
		\paragraph{}
			L'inspection du travail a elle aussi un rôle à jouer en dénonçant les abus dans les entreprises. Cependant, le faible nombre d'inspecteurs du travail ne facilite pas cette tâche.
		\paragraph{}
			\textit{``Selon le Rapport d’activité de l’Inspection du travail, il y avait en France en 2008 535 inspecteurs du travail et 1 171 contrôleurs chargés de contrôler 1 600 000 entreprises regroupant 16 millions de salariés.
			Soit un inspecteur ou contrôleur pour 1 000 entreprises.''} - Article Viva Presse
		%- les problèmes ne se règle plus à l'intérieur mais avec la justice
		%- l'inspection du travail qui doit permettre de lutter contre la souffrance au travail (denoncer, déclarer)
	\section{Quels traitements ?}
		\paragraph{}
			En plus d'être légalement en droit de se retourner contre l'entreprise, le salarié dispose aussi de plusieurs moyens pour traiter sa souffrance. En ayant recours aux médecins et psychologues, qu'ils soient internes ou externes à l'entreprise, les personnes peuvent améliorer leur état physique et mental. Cependant, ces recours à la médecine reflètent assez bien l'état de souffrance d'un pays : par exemple, les français sont les premiers consommateurs d'anti-dépresseurs au monde. Ce serait en effet plus d'un français sur 4 qui serait sous psychotrope.\\
			Même s'il est important de relativiser l'utilisation des anti-dépresseurs, produits plus commerciaux que véritables traitements, cela est quand même un bon indicateur de la souffrance psychologique des français.
		%- trouble muscolo squeletique, stress, 
		%- francais premier consomateur d'antidéprésseur
		%- traitement, méditation, yoga
		%- installation medecin/psychologue d'entreprise = banalisation de la souffrance
	\section{Peut-on éviter la souffrance au travail ?}
		\paragraph{}
			La piste la plus intéressante au niveau de la gestion de la souffrance au travail est quand même d'essayer de la diminuer au maximum, visant même l'utopie d'éviter les souffrances.
		\paragraph{}
			Pour cela, les entreprises ont prit depuis assez longtemps bon nombre de mesures pour augmenter la sécurité dans les usines et ainsi, même si les employés sont, comme dit précédemment, parfois amenés à outrepasser les règles de sécurité volontairement, diminuer le nombre d'accident du travail. L'amélioration des postes de travail est, quant à elle, une manière d'essayer de réduire au maximum les troubles musculo-squelettiques.
		\paragraph{}
			\begin{center}
				\includegraphics[scale=0.3]{graph1.png}
			\end{center}
		\paragraph{}
			Depuis les années 90, le stress a commencé à entrer dans les considérations par rapport à la souffrance au travail. Ainsi, les anciennes formes de management basées sur la pression et la peur sont de moins en moins enseignées dans les écoles de management pour laisser place à des formes de management plus respectueuse des personnes, basées sur la compréhension des besoins du salarié et sur la gratification du travail effectué. 
    	%- nouvelle forme de management (plus autour de l'écoute du salarié, plus humain)
\chapter*{Conclusion}
\addcontentsline{toc}{chapter}{Conclusion}
    \paragraph*{}
         
\end{document}