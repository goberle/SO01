\documentclass{report}
\usepackage[utf8]{inputenc}
\usepackage[T1]{fontenc}
\usepackage[francais]{babel}
\usepackage[final]{pdfpages}

\title{Sociologie du travail\\Comment la souffrance au travail a-t-elle évoluée du capitalisme industriel au capitalisme financier ?}
\author{Vincent Chabrette\\Guillaume Oberlé}
\date{18 Décembre 2013}

\begin{document}

\makeatletter
\def\clap#1{\hbox to 0pt{\hss #1\hss}}%
\def\ligne#1{%
\hbox to \hsize{%
\vbox{\centering #1}}}%
\def\haut#1#2#3{%
\hbox to \hsize{%
\rlap{\vtop{\raggedright #1}}%
\hss
\clap{\vtop{\centering #2}}%
\hss
\llap{\vtop{\raggedleft #3}}}}%
\def\bas#1#2#3{%
\hbox to \hsize{%
\rlap{\vbox{\raggedright #1}}%
\hss
\clap{\vbox{\centering #2}}%
\hss
\llap{\vbox{\raggedleft #3}}}}%

\def\maketitle{%
\thispagestyle{empty}\vbox to \vsize{%
\haut{}{\@blurb}{}
\vfill
\vspace{1cm}
\begin{flushleft}
\usefont{OT1}{ptm}{m}{n}
\huge \@title
\end{flushleft}
\hrule height 4pt
\begin{flushright}
\usefont{OT1}{phv}{m}{n}
\large \@author
\end{flushright}
\begin{flushleft}
\end{flushleft}
\vspace{1cm}
\vfill
\begin{center}
\includegraphics[width=.5\textwidth]{./logo.png}
\end{center}
\vfill
\vfill
\bas{}{\@location, le \@date}{}
}%
\cleardoublepage
}
\def\date#1{\def\@date{#1}}
\def\author#1{\def\@author{#1}}
\def\title#1{\def\@title{#1}}
\def\location#1{\def\@location{#1}}
\def\blurb#1{\def\@blurb{#1}}
\makeatother

\title{Comment la souffrance au travail a-t-elle évoluée du capitalisme industriel au capitalisme financier ?}
\author{Vincent \textsc{Chabrette}\\Guillaume \textsc{Oberlé}}
\location{Belfort}
\blurb{%
Sociologie du travail\\Université de Technologie de Belfort-Montbeliard}
\maketitle
\tableofcontents

\chapter*{Introduction}
\addcontentsline{toc}{chapter}{Introduction}
	\paragraph{}
		Dans le cadre de l'UV SO01, Sociologie du travail, nous avons choisi, pour l'évaluation de la matière, de traiter un sujet sur le thème de la souffrance au travail. Nous avons décidé de nous intéresser plus particulièrement aux évolutions de la souffrance au travail. Ce rapport aura donc pour objectif de répondre à la problématique \textit{Comment la souffrance au travail a-t-elle évoluée du capitalisme industriel au capitalisme financier ?}

	\paragraph{}
		L’objectif sera d’analyser les différences entre la souffrance au travail dans le cadre du capitalisme industriel des années 60 et celle dans le cadre du capitalisme financier d’aujourd’hui. \\Ce sujet nous a été inspiré par le film ``J’ai très mal au travail'' de Jean-Michel Carré qui constituera aussi une de nos sources principales d’information.

	\paragraph{}
		Pour répondre à cette problématique, nous avons divisé notre rapport en quatre grandes parties. Tout d'abord nous nous intéresserons à la souffrance dans le cadre industriel. Nous essayerons de montrer qu'elle se traduit inéluctablement par une souffrance physique, mais aussi par une souffrance psychique non négligeable. Dans une deuxième partie, nous étudierons les différents aspects de la souffrance au travail dans notre société actuelle, majoritairement dominée par le capitalisme financier. Puis, dans une troisième partie, nous nous poserons la question de l'acceptation et de la banalisation de la souffrance au travail. Enfin, dans une quatrième et dernière partie, nous énoncerons ce qu'il est possible de faire, et ce qui fait à l'heure actuelle, pour contrer et traiter la souffrance au travail.

\chapter{La souffrance dans le cadre industriel : une souffrance physique ?}
	\section{Les manifestations de la souffrance au travail}
		\paragraph{}
			Le capitalisme industriel, qui a vu son apparition au début des années 1960 se caractérise par un productivisme nécessitant beaucoup d'ouvriers travaillant à la chaîne. En effet, c'est dans ces années là que nos sociétés modernes ont vu apparaître de nouvelles formes de consommations basé sur la création perpétuelle de nouveau besoin. Ces besoins créé par les grandes entreprises nécessite donc d'importante force de travail pour permettre une production de masse, ce qui impose aux chaînes de production des cadences parfois inhumaine, notamment du à l'organisation scientifique du travail (Taylorisme/Fordisme).

		\paragraph{}
			De ce fait, le capitalisme industriel est fortement créateur chez ces travailleurs de troubles musculo-squelettique, d'accident du travail et autres diverses pathologie physiques.\\
			Ces souffrances physiques sont ... imputables aux gestes répétés, aux cadences de travail, mais aussi assez souvent à cause du salarié lui-même qui va être amené à négliger les règles de sécurité pour pouvoir répondre à la pression qui leur est imposé. 

	\section{Une souffrance psychique non négligeable}

\chapter{La souffrance au travail dans notre société actuelle}
	\section{La souffrance au travail dans le cadre du capitalisme financier}

	\section{Des outils de gestions créateurs de souffrance psychique}

	\section{La productivité au détriment de l'humain}

	\section{Comparaison avec la souffrance physique}

\chapter{L’acceptation de la souffrance : la souffrance banalisée ?}
	\section{Le rapport du salarié au travail}

	\section{La concurrence dans le monde du travail}

	\section{La valeur de l'emploi}

\chapter{La souffrance au travail : une fatalité ?}
	\section{La lutte contre la souffrance au travail}

	\section{Quel traitement ?}

	\section{Peut-on éviter la souffrance au travail ?}
    
\chapter*{Conclusion}
\addcontentsline{toc}{chapter}{Conclusion}
    \paragraph*{}
         
\end{document}