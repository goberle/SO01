\documentclass{report}
\usepackage[utf8]{inputenc}
\usepackage[T1]{fontenc}
\usepackage[francais]{babel}
\usepackage[final]{pdfpages}

\title{Sociologie du travail\\Comment la souffrance au travail a-t-elle évoluée du capitalisme industriel au capitalisme financier ?}
\author{Vincent Chabrette\\Guillaume Oberlé}
\date{18 Décembre 2013}

\begin{document}

\makeatletter
\def\clap#1{\hbox to 0pt{\hss #1\hss}}%
\def\ligne#1{%
\hbox to \hsize{%
\vbox{\centering #1}}}%
\def\haut#1#2#3{%
\hbox to \hsize{%
\rlap{\vtop{\raggedright #1}}%
\hss
\clap{\vtop{\centering #2}}%
\hss
\llap{\vtop{\raggedleft #3}}}}%
\def\bas#1#2#3{%
\hbox to \hsize{%
\rlap{\vbox{\raggedright #1}}%
\hss
\clap{\vbox{\centering #2}}%
\hss
\llap{\vbox{\raggedleft #3}}}}%

\def\maketitle{%
\thispagestyle{empty}\vbox to \vsize{%
\haut{}{\@blurb}{}
\vfill
\vspace{1cm}
\begin{flushleft}
\usefont{OT1}{ptm}{m}{n}
\huge \@title
\end{flushleft}
\hrule height 4pt
\begin{flushright}
\usefont{OT1}{phv}{m}{n}
\large \@author
\end{flushright}
\begin{flushleft}
\end{flushleft}
\vspace{1cm}
\vfill
\begin{center}
\includegraphics[width=.5\textwidth]{./logo.png}
\end{center}
\vfill
\vfill
\bas{}{\@location, le \@date}{}
}%
\cleardoublepage
}
\def\date#1{\def\@date{#1}}
\def\author#1{\def\@author{#1}}
\def\title#1{\def\@title{#1}}
\def\location#1{\def\@location{#1}}
\def\blurb#1{\def\@blurb{#1}}
\makeatother

\title{Comment la souffrance au travail a-t-elle évoluée du capitalisme industriel au capitalisme financier ?}
\author{Vincent \textsc{Chabrette}\\Guillaume \textsc{Oberlé}}
\location{Belfort}
\blurb{%
Sociologie du travail\\Université de Technologie de Belfort-Montbeliard}
\maketitle
\tableofcontents

\chapter*{Introduction}
\addcontentsline{toc}{chapter}{Introduction}
	\paragraph{}
		Dans le cadre de l'UV SO01, Sociologie du travail, nous avons choisi, pour l'évaluation de la matière, de traiter un sujet en relation avec la souffrance au travail. Nous avons décidé de s'intéresser plus particuliérement sur les évolutions de la souffrance au travail. Ce rapport aura donc pour objectif de répondre à la question \textit{Comment la souffrance au travail a-t-elle évoluée du capitalisme industriel au capitalisme financier ?}

	\paragraph{}
		L’objectif sera d’analyser les différences entre la souffrance au travail dans le cadre du capitalisme industriel des années 60 et celle dans le cadre du capitalisme financier d’aujourd’hui. Ce sujet nous a été inspiré par le film J’ai très mal au travail de Jean-Michel Carré qui constituera aussi une de nos sources principales d’information.

	\paragraph{}
		Pour répondre à cette problématique, nous avons diviser notre rapport en quatre grande partie. Tout d'abord nous nous intéresserons à la souffrance dans le cadre industriel. Nous essayerons de montrer qu'elle se traduit majoritairement par une souffrance physique. Puis dans une deuxième partie, nous essayerons de voir les différents aspects de la souffrance au travail dans notre société actuelle. Dans une troisième partie, nous nous poserons la question de l'acceptation et de la banalité de la souffrance au travail. Enfin, dans une dernière partie, nous énoncerons ce qui est possible et ce qui fait à l'heure d'aujourd'hui pour contrer et traiter la souffrance au travail.

\chapter{La souffrance dans le cadre industriel : une souffrance physique ?}

\chapter{La souffrance au travail dans notre société actuelle}

\chapter{L’acceptation de la souffrance : la souffrance banalisée ?}

\chapter{La souffrance au travail : une fatalité ?}
    
\chapter*{Conclusion}
\addcontentsline{toc}{chapter}{Conclusion}
    \paragraph*{}
         
\end{document}