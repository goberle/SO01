\documentclass{report}
\usepackage[utf8]{inputenc}
\usepackage[T1]{fontenc}
\usepackage[francais]{babel}
\usepackage[final]{pdfpages}

\title{Sociologie du travail\\Comment la souffrance au travail a-t-elle évoluée du capitalisme industriel au capitalisme financier ?}
\author{Vincent Chabrette\\Guillaume Oberlé}
\date{18 Décembre 2013}

\begin{document}

\makeatletter
\def\clap#1{\hbox to 0pt{\hss #1\hss}}%
\def\ligne#1{%
\hbox to \hsize{%
\vbox{\centering #1}}}%
\def\haut#1#2#3{%
\hbox to \hsize{%
\rlap{\vtop{\raggedright #1}}%
\hss
\clap{\vtop{\centering #2}}%
\hss
\llap{\vtop{\raggedleft #3}}}}%
\def\bas#1#2#3{%
\hbox to \hsize{%
\rlap{\vbox{\raggedright #1}}%
\hss
\clap{\vbox{\centering #2}}%
\hss
\llap{\vbox{\raggedleft #3}}}}%

\def\maketitle{%
\thispagestyle{empty}\vbox to \vsize{%
\haut{}{\@blurb}{}
\vfill
\vspace{1cm}
\begin{flushleft}
\usefont{OT1}{ptm}{m}{n}
\huge \@title
\end{flushleft}
\hrule height 4pt
\begin{flushright}
\usefont{OT1}{phv}{m}{n}
\large \@author
\end{flushright}
\begin{flushleft}
\end{flushleft}
\vspace{1cm}
\vfill
\begin{center}
\includegraphics[width=.5\textwidth]{./logo.png}
\end{center}
\vfill
\vfill
\bas{}{\@location, le \@date}{}
}%
\cleardoublepage
}
\def\date#1{\def\@date{#1}}
\def\author#1{\def\@author{#1}}
\def\title#1{\def\@title{#1}}
\def\location#1{\def\@location{#1}}
\def\blurb#1{\def\@blurb{#1}}
\makeatother

\title{Comment la souffrance au travail a-t-elle évoluée du capitalisme industriel au capitalisme financier ?}
\author{Vincent \textsc{Chabrette}\\Guillaume \textsc{Oberlé}}
\location{Belfort}
\blurb{%
Sociologie du travail\\Université de Technologie de Belfort-Montbeliard}
\maketitle
\tableofcontents

\chapter*{Introduction}
\addcontentsline{toc}{chapter}{Introduction}
	\paragraph{}
		Dans le cadre de l'UV SO01, Sociologie du travail, nous avons choisi, pour l'évaluation de la matière, de traiter un sujet sur le thème de la souffrance au travail. Nous avons décidé de nous intéresser plus particulièrement aux évolutions de la souffrance au travail. Ce rapport aura donc pour objectif de répondre à la problématique \textit{Comment la souffrance au travail a-t-elle évoluée du capitalisme industriel au capitalisme financier ?}

	\paragraph{}
		L’objectif sera d’analyser les différences entre la souffrance au travail dans le cadre du capitalisme industriel des années 60 et celle dans le cadre du capitalisme financier d’aujourd’hui. \\Ce sujet nous a été inspiré par le film ``J’ai (très) mal au travail'' de Jean-Michel Carré qui constituera aussi une de nos sources principales d’information.

	\paragraph{}
		Pour répondre à cette problématique, nous avons divisé notre rapport en quatre grandes parties. Tout d'abord nous nous intéresserons à la souffrance dans le cadre industriel. Nous essayerons de montrer qu'elle se traduit inéluctablement par une souffrance physique, mais aussi par une souffrance psychique non négligeable. Dans une deuxième partie, nous étudierons les différents aspects de la souffrance au travail dans notre société actuelle, majoritairement dominée par le capitalisme financier. Puis, dans une troisième partie, nous nous poserons la question de l'acceptation et de la banalisation de la souffrance au travail. Enfin, dans une quatrième et dernière partie, nous énoncerons ce qu'il est possible de faire, et ce qui fait à l'heure actuelle, pour contrer et traiter la souffrance au travail.

\chapter{La souffrance dans le cadre industriel : une souffrance physique ?}
	\section{Les manifestations de la souffrance au travail}
		\paragraph{}
			Le capitalisme industriel, qui a vu son apparition au début des années 1960 se caractérise par un productivisme nécessitant beaucoup d'ouvriers travaillant à la chaîne. En effet, c'est dans ces années là que nos sociétés modernes ont vu apparaître de nouvelles formes de consommations basé sur la création perpétuelle de nouveau besoin. Ces besoins créé par les grandes entreprises nécessite donc d'importante force de travail pour permettre une production de masse, ce qui impose aux chaînes de production des cadences parfois inhumaine, notamment du à l'organisation scientifique du travail (Taylorisme/Fordisme).

		\paragraph{}
			De ce fait, le capitalisme industriel est fortement créateur chez ces travailleurs de troubles musculo-squelettique, d'accident du travail et autres diverses pathologie physiques.\\
			Ces souffrances physiques sont ... imputables aux gestes répétés, aux cadences de travail, mais aussi assez souvent à cause du salarié lui-même qui va être amené à négliger les règles de sécurité pour pouvoir répondre à la pression qui leur est imposé. 

	\newpage{}
	\section{Une souffrance psychique non négligeable}
		\paragraph{}
			Même si la souffrance physique est la première venant à l'esprit lorsque l'on parle de capitalisme industriel, celui-ci est aussi créateur de souffrance psychique. À l'instar de la souffrance physique, la souffrance psychique est ... aux cadences et à la pression imposé par le travail ouvrier. A noter aussi qu'une partie de celle-ci est imputable à la souffrance physique. Cependant une grande partie de cette souffrance psychique était contre balancé par l'esprit de camaraderie, les dirigeants paternalistes, etc...

		\paragraph{}
			En revanche depuis la perte de vitesse du capitalisme industriel face au capitalisme financier, à cause de la concurrence omniprésente ainsi que de l'instabilité du marché du travail, le travail ouvrier tend à partager avec les grandes entreprises de services certains de leur éléments de souffrance psychique. En effet, les relations au travail se détériorent aussi et l'ouvrier vie dans la peur constante de l'avenir qui est, comme la conservation de leur emploi, baigné d'incertitudes.

\chapter{La souffrance au travail dans notre société actuelle}
	\paragraph{}
		\textit{``Chaque année 2 millions de salariés subissent maltraitance et harcèlement moral''} - Tiré du film ``J’ai (très) mal au travail'' de Jean-Michel Carré.
	\section{La souffrance au travail dans le cadre du capitalisme financier}
		\paragraph{}
			Dans nos sociétés modernes soumises à la mondialisation, le marché du travail des pays développés a vu l'importance des secteurs primaires et secondaires diminuer face au secteur tertiaire. Ajouté à cela des sociétés toujours plus basées sur la consommation, on se rend alors compte que cela crée un capitalisme principalement tourné vers le profit : c'est ce que l'on appelle communément le capitalisme financier. Il se différencie du capitalisme industriel par le fait qu'il crée énormément de richesse avec peu de moyens humains là ou ce dernier créé un peu moins de richesses en nécessitant beaucoup plus de travailleurs.
		\paragraph{}
			Il parait évident que ce capitalisme n'est pas principalement créateur de souffrance physique. En revanche, la recherche du profit est fortement créatrice de souffrance psychique. En effet, cette course au profit implique une modification profonde du rapport et de l'organisation au travail.

	\section{Des outils de gestions créateurs de souffrance psychique}
		\paragraph{}
			De part la forte concurrence au sein du monde du travail, l'entreprise se doit d'être réactive et rapide. Le manager doit être en mesure de chiffrer les résultats des personnes qu'il encadre et répercute la pression qu'il reçoit aux salariés dont il a la charge; en partie involontairement, mais aussi et surtout car le management par la pression et la peur est une méthode managériale qui fut longtemps prônée comme la plus efficace. Cette technique est donc logiquement fortement créatrice de stress et de mal-être chez le salarié.
		\paragraph{}
		 	Ainsi, pour "aider" l'employé à faire face à cette pression, on assiste de plus en plus à la création de séminaires, notamment pour les cadres, visant à les remotiver, à augmenter la cohésion d'équipe, à gérer leur stress, etc... Leurs employeurs s'assurent ainsi (et surtout) de créer une culture d'entreprise et de lier les salariés entre eux, les impliquant encore plus dans la société. Certaines entreprises vont, par exemple, embaucher des entraîneurs de rugby qui organiseront des activités ludiques durant lesquelles la hiérarchie n'a plus d'importance (ou n'est, du moins, plus censée) et où seul l'objectif imposé compte, exercice étant censé rapprocher les managers des managés, mais qui au final visera surtout à humaniser les relations entre eux pour que le managé accepte plus facilement ce que lui demande son manager. 
		\paragraph{}
			L'arrivée des nouvelles technologies de l'information et de la communication a elle aussi permit aux managers de trouver de nouvelles façons de conserver constamment les salariés au service de l'entreprise en créant un lien permanent entre l'employé et son employeur. \\La démocratisation de la téléphonie mobile et de l'Internet fit entrer le monde de l'entreprise dans une époque de l'urgence et de l'instantanéité. Ainsi, les employeurs n'hésitent plus à fournir un téléphone mobile ainsi qu'un ordinateur portable à chaque employé. L'employé doit alors être disponible, en tout temps et à toute heure, pour son entreprise.
		\paragraph{}
			La notation des employés est également un outil de gestion supplémentaire qui s'offre au manager. La technique communément utilisée est celle de l'autoévaluation où l'on va demander à l'employé sont avis sur la notation que vient de faire son supérieur, et ainsi permettre de faire participer le salarié à son évaluation de sorte qu'il accepte plus facilement les conséquences, qu'elles soit positives ou négatives.
		\paragraph{}
			Ces techniques managériales ne sont que des exemples parmi tant d'autres, cependant la perversité de ces nouvelles formes de management commence dès l'embauche; en effet, on n'embauche plus les gens uniquement sur leurs compétences mais aussi sur leur capacité à se soumettre à la hiérarchie et à rentrer facilement dans le moule de l'entreprise. On va alors privilégier des personnes ayant beaucoup à perdre s'il perdaient leur emplois (personne avec famille, prêts à rembourser, etc...)
		\paragraph{}
			Via ce nouveau management, l'entreprise prend de plus en plus le pas sur la vie privée des salariés : l'employé ne peut alors cesser de "penser entreprise" que cela soit chez lui ou même lors de la pause café. Cela ancre alors profondément l'entreprise dans la vie de la personne.

		%- concurrence forte au sein du monde du travail
		%  -> l'entreprise se doit d'être réactive, rapide
		%  -> on demande au manager plus de résultat
		%  -> manager répercute la pression sur les employés
		%- nouvelle technologie, l'employé est lié à l'entreprise
		%  -> époque de l'urgence et de l'instantanéité
		%- on embauche des gens qui se soumettent au détriment des compétences
		%  -> gens qui ont des choses à perdre
		%- organisation de séminaire pour créer une culture d'entreprise et lié les salariés entre eux
		%- notation des employées
		%  -> autoévaluation : implique dans la notation et lui fait accepter plus facilement les conséquences positives et négative
		%l'entreprise percute la vie privé des salariés
		%  -> les gens pense en permanence entreprise (pause café, soir, etc..)
		%- beauté du geste : salarié expérimenté en face de nouveau
		%* But: impliquer les salariés dans l'entreprise et dans leur travail (en va de la vie de l'entreprise, cf: Renault)		

	\section{La productivité au détriment de l'humain}
		\paragraph{}
			Pour survivre dans un marché hautement concurrentiel, l'entreprise va, en plus d'opérer une refonte de son management (comme vu précédemment), chercher tous les moyens possibles pour augmenter sa productivité. 
		\paragraph{}
			L'entreprise peut alors opérer des changements de postes par exemple. Ainsi, dans le film "J'ai (très) mal au travail", nous avons pu découvrir le cas d'une ouvrière dans le secteur textile qui travaillait dans un atelier de tissage et qui a été déplacée à un autre poste moins valorisant. N'ayant plus le sentiment d'effectuer un travail important et utile, cette ouvrière fut désespérée de sa situation et à en souffrir profondément, la rendant même prête à se faire mal pour pouvoir obtenir un arrêt de travail. On se rend alors compte que cette décision de la direction, prise sans aucune considération de la volonté propre de la salariée, créa elle aussi une véritable souffrance psychique pour elle.
		\paragraph{}
			Avec la mondialisation, les sociétés ont aussi la possibilité de délocaliser leurs usines pour diminuer leurs coûts de production. Il se peut aussi qu'en cas de grande difficulté financière (ou d'un patron abusant du système pour faire un maximum de profit, au détriment de la survie de la société) l'entreprise soit amenée à fermer des lieux de travail, qu'il s'agisse d'usines ou de bureaux. Dans ces 2 cas, il y a un dénominateur commun : le licenciement de masse.\\
			Acte quasi-abstrait pour les dirigeants d'entreprise, ces licenciements marquent en revanche pour de nombreux salariés (qu'il s'agisse de dizaines ou de milliers) le basculement de leur vie. C'est en effet à ce moment là que les (ex-)salariés vont prendre conscience de la place qu'avait prise l'entreprise dans leur vie. Ainsi, lors de fermeture de sites de productions par exemple, nombreuses sont les personnes sans diplôme, d'un âge déjà relativement avancé et ayant consacré des dizaines d'années de leur vie à l'entreprise, qui vont se retrouver sans emploi, mais aussi sans statut social et sans une partie de leur identité. Nombreuses sont alors les remises en question, mais aussi les soucis d'argent dû à la difficulté de retrouver un emploi, qui montrent bien un des grands pouvoirs de l'entreprise : un quasi-pouvoir de vie et de mort sur ses salariés.

		%- dans le secteur industriel pour produire plus à plus bas coût les entreprises délocalise et/ou licencie
		%- exemple : changement de poste au détriment de la satisfaction du salarié
		%- relation de vie et de mort avec l'entreprise
		%- productivité en dépit du rapport humain
		%  -> chacun à son poste
		%  -> pas de rapport humain

	\section{Comparaison avec la souffrance physique}
		\paragraph{}
			%http://www.meteojob.com/emploi-actualites/actualite/158/la-souffrance-psychique-au-travail-un-mal-plus-courant-quon-croit
			%http://www.intefp-sstfp.travail.gouv.fr/datas/files/SSTFP/Souffrance_mentale_au_travail%20INTEFP%202005.pdf => (p7) http://dpr.vchabrette.fr/O3xZ

\chapter{L’acceptation de la souffrance : la souffrance banalisée ?}
	\section{Le rapport du salarié au travail}
		- fierté par rapport a son travail
			-> le salarié souffre mais le fait de se sentir impliquer dans son travail et surtout de recevoir de la reconnaissance de ses pairs et/ou de ses supérieurs font qu'il est fier
			-> exemple de cas : ouvrier a la chaîne reconnaisse travail de l'autre
			-> reconnaissance est une des chose les plus importantes cherché par le salarié
	\section{La concurrence dans le monde du travail}
		- concurrence entre les salariés, si on veut être le meilleur, pas le temps de souffrir
		- la mondialisation met en concurrence les salariés entre salarié du même pays, voir même du monde 
	\section{La valeur de l'emploi}
		- marché du travail incertain = pas sur de retrouver un emploi
		- peur de perdre notre emploi
		- impression d'avoir quelque chose que tout le monde n'a pas, fierté d'avoir un travail
		- risque de tout perdre fait accepter n'importe quoi
		- l'entreprise représente ta vie
		- a paris 1 sdf sur 5 a du travail
		- risque de chomage fait accepter de souffrir / de faire souffrir

	* Cela crée une déhumanisation de l'homme et une servitude volontaire
	* Ce besoin d'avoir de l'argent est imposé par la société de consommation
\chapter{La souffrance au travail : une fatalité ?}
	\section{La lutte contre la souffrance au travail}
		- les problèmes ne se règle plus à l'intérieur mais avec la justice
		- l'inspection du travail qui doit permettre de lutter contre la souffrance au travail (denoncer, déclarer)
	\section{Quel traitement ?}
		- trouble muscolo squeletique, stress, 
		- francais premier consomateur d'antidéprésseur
		- traitement, méditation, yoga
		- installation medecin/psychologue d'entreprise = banalisation de la souffrance
	\section{Peut-on éviter la souffrance au travail ?}
    	- nouvelle forme de management (plus autour de l'écoute du salarié, plus humain)
\chapter*{Conclusion}
\addcontentsline{toc}{chapter}{Conclusion}
    \paragraph*{}
         
\end{document}